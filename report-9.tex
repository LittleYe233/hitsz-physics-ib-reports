\documentclass[signature=data]{physicsreport}

%%
%% User settings
%%

\classno{}
\stuno{}
\groupno{}
\stuname{}
\expdate{\expdatefmt\today}
\expname{弗兰克-赫兹实验}

%%
%% Document body
%%

\begin{document}
% First page
% Some titles and personal information are defined in ``\maketitle''.
\maketitle

\section{实验预习}
\begin{enumerate}
    \item 简要叙述波尔的原子能级理论;
    \item 描述弗兰克-赫兹的实验原理.
\end{enumerate}

% Teacher signature
\makeatletter
\physicsreport@body@signature{preparation}
\makeatother

\newpage
% Original experiment data
\section{实验现象及原始数据记录}

% Teacher signature
\makeatletter
\physicsreport@body@signature{data}
\makeatother

\newpage
% Data process and others
\section{数据处理}
\begin{enumerate}
    \item 利用计算机软件绘制 $I_A-U_{G2K}$ 曲线;
    \item 对曲线进行拟合, 利用各峰值或波谷所对应的电压值, 分别用逐差法和最小二乘法计算氩原子的第一激发电位.
\end{enumerate}

% This vertical gap is for handwriting use. It can be removed when typing answers here.
\vspace*{30em}
\section{实验结论及现象分析}

\newpage
\section{讨论题}
\begin{enumerate}
    \item 在 $I_A-U_{G2K}$ 曲线中, 为什么随着 $U_{G2K}$ 的增大, 波谷电流逐渐增大?
    \item 请分析拒斥电压改变对 $I_A-U_{G2K}$ 曲线的影响.
    \item 为什么弗兰克-赫兹实验只能测出第一激发态电位?
\end{enumerate}

\end{document}
