\documentclass[signature=preparation]{physicsreport}

%%
%% User settings
%%

\classno{}
\stuno{}
\groupno{}
\stuname{}
\expdate{\expdatefmt\today}
\expname{自组显微镜与望远镜}

%%
%% Document body
%%

\begin{document}
% First page
% Some titles and personal information are defined in ``\maketitle''.
\maketitle

\section{预习}
\begin{enumerate}
    \item 请分别绘制出显微镜和望远镜的光路图.
    \item 结合光路图, 请分别推导显微镜和望远镜放大率的计算公式.
\end{enumerate}

% Teacher signature
\makeatletter
\physicsreport@body@signature{preparation}
\makeatother

\newpage
% Original experiment data
\section{原始数据记录}
\subsection{自组显微镜放大率测量}
物镜 $L_o$ ($f_0'=45\ \mathrm{mm}$) 目镜 $L_e$ ($f_e'=34\ \mathrm{mm}$) $L=25\ \mathrm{cm}$

\begin{table*}[ht]
    \renewcommand{\arraystretch}{1.4}
    \small\selectfont
    \centering
    \begin{tabularx}{\textwidth}{|c|Y|Y|Y|Y|Y|Y|Y|}\hline
        序号 & 物镜 $L_o$ 位置 (mm) & 目镜 $L_e$ 位置 (mm) & 分划板 $M_1$ 位置 (mm) & 标尺 $M_2$ 位置 (mm) & 光学筒长 $\Delta$ (mm) & $M_2$ 标尺中距离 $d$ (mm) & 对应 $M_1$ 格数 $a$ \\\hline
        1  &                  &                  &                   &                  &                    &                      &                 \\\hline
        2  &                  &                  &                   &                  &                    &                      &                 \\\hline
        3  &                  &                  &                   &                  &                    &                      &                 \\\hline
        4  &                  &                  &                   &                  &                    &                      &                 \\\hline
        5  &                  &                  &                   &                  &                    &                      &                 \\\hline
    \end{tabularx}
\end{table*}

\subsection{自组望远镜放大率测量}
物镜 $L_o$ ($f_0'=225\ \mathrm{mm}$) 目镜 $L_e$ ($f_e'=45\ \mathrm{mm}$) $L=25\ \mathrm{cm}$

\begin{table*}[ht]
    \renewcommand{\arraystretch}{1.65}
    \small\selectfont
    \centering
    \begin{tabularx}{\textwidth}{|c|Y|Y|Y|Y|Y|}\hline
        序号 & 物镜 $L_o$ 位置 (mm) & 目镜 $L_e$ 位置 (mm) & 标尺距离物镜的距离 (mm) & 红色指针距离 $d_1$ (mm) & 直观标尺长度 $d_2$ (mm) \\\hline
        1  &                  &                  &                &                   &                   \\\hline
        2  &                  &                  &                &                   &                   \\\hline
        3  &                  &                  &                &                   &                   \\\hline
        4  &                  &                  &                &                   &                   \\\hline
        5  &                  &                  &                &                   &                   \\\hline
    \end{tabularx}
\end{table*}

% Teacher signature
\makeatletter
\physicsreport@body@signature{data}
\makeatother

\newpage
% Data process and others
\section{数据处理}
\begin{enumerate}
    \item 分别求出自组显微镜测量放大率和计算放大率.
    \item 分别求出自组望远镜实际测量放大率和无限远放大率.
\end{enumerate}

% This vertical gap is for handwriting use. It can be removed when typing answers here.
\vspace*{15em}
\section{实验现象分析及结论}

\vspace*{12em}
\section{讨论题}
\begin{enumerate}
    \item 请简述显微镜与望远镜的区别?
    \item 请思考自组望远镜实际视放大率测量值与无限远放大率数值出现差异的原因?
\end{enumerate}

\end{document}
