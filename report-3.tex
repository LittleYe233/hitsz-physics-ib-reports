\documentclass[signature=data]{physicsreport}

%%
%% User settings
%%

\classno{}
\stuno{}
\groupno{}
\stuname{}
\expdate{\expdatefmt\today}
\expname{光的等厚干涉现象与应用}

%%
%% Document body
%%

\begin{document}
% First page
% Some titles and personal information are defined in ``\maketitle''.
\maketitle

\section{预习}
预习指导书, 设牛顿环的第 $m$ 级暗环半径为 $r_m$, 该处对应的空气隙厚度为 $d$, 凸透镜的凸面曲率半径为 $R$, 空气隙折射率取 $n=1$, 推导出牛顿环的第 $m$ 级暗环半径 $r_m$ 的表达式 $r_m=\sqrt{m\lambda\left(R-\dfrac{m\lambda}{4}\right)}$.

% Teacher signature
\makeatletter
\physicsreport@body@signature{preparation}
\makeatother

\newpage
% Original experiment data
\section{原始数据记录}
\subsection{牛顿环测透镜曲率半径数据记录}
\begin{table*}[ht]
     \centering
     \begin{tabularx}{\textwidth}{|>{\bfseries}c|>{\bfseries}c|Y|Y|Y|Y|Y|Y|} \hline
          环的序数 & $m$ & 31 & 30 & 29 & 28 & 27 & 26 \\\hline
          \multirow{2}{*}{\makecell{环的位置           \\读数 / mm}} & 左 & & & & & & \\\cline{2-8}
               & 右   &    &    &    &    &    &    \\\hline
          环的序数 & $m$ & 25 & 24 & 23 & 22 & 21 &    \\\hline
          \multirow{2}{*}{\makecell{环的位置           \\读数 / mm}} & 左 & & & & & & \\\cline{2-8}
               & 右   &    &    &    &    &    &    \\\hline
          环的序数 & $n$ & 20 & 19 & 18 & 17 & 16 & 15 \\\hline
          \multirow{2}{*}{\makecell{环的位置           \\读数 / mm}} & 左 & & & & & & \\\cline{2-8}
               & 右   &    &    &    &    &    &    \\\hline
          环的序数 & $n$ & 14 & 13 & 12 & 11 & 10 &    \\\hline
          \multirow{2}{*}{\makecell{环的位置           \\读数 / mm}} & 左 & & & & & & \\\cline{2-8}
               & 右   &    &    &    &    &    &    \\\hline
     \end{tabularx}
\end{table*}

\subsection{劈尖干涉测磁带厚度数据记录}
\begingroup
\noindent\centering\begin{tabularx}{\textwidth}{|c|Y|Y|} \hline
     \bfseries 测量次数 & \bfseries 第 $i$ 条干涉条纹位置 $x_1$ (mm) & \bfseries 第 $(i+10)$ 条干涉条纹位置 $x_2$ (mm) \\\hline
     1              &                                    &                                         \\\hline
     2              &                                    &                                         \\\hline
     3              &                                    &                                         \\\hline
     4              &                                    &                                         \\\hline
     5              &                                    &                                         \\\hline
\end{tabularx}
\endgroup

% Teacher signature
\makeatletter
\physicsreport@body@signature{data}
\makeatother

\newpage
% Data process and others
\section{数据处理}
用逐差法求 $D_m^2-D_n^2$ 的平均值; 计算曲率半径 $R$ 的平均值及不确定度; 计算磁带的厚度, 要有完整的计算过程.

\newpage
\section{实验结论及现象分析}

\vspace*{8cm}
\section{讨论题}
\begin{enumerate}
     \item 理论上牛顿环中心是个暗点, 实际上看到的往往是个忽明忽暗的斑, 其原因是什么? 对透镜曲率半径 $R$ 测量有无影响?
     \item 实验中, 若平板玻璃上有微小的凸起, 则凸起处的干涉条纹会发生如何变化?
\end{enumerate}

\end{document}
